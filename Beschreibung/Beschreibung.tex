\documentclass{scrartcl}

\usepackage{lmodern}
\usepackage[ngerman]{babel}
\usepackage[ansinew]{inputenc}
\usepackage[T1]{fontenc}


\usepackage{graphicx}
\usepackage{picinpar}
\usepackage[colorlinks=true,linkcolor=black]{hyperref}

%Source Code
\usepackage{color,listings}                %bindet das Paket Listings ein
\definecolor{comment}{rgb}{.15,.4,.15}     % hellgruen
\definecolor{keywd1}{rgb}{.15,.15,.6}      % dunkelblau
\definecolor{keywd2}{rgb}{.35,.5,.55}      % hellblau
\definecolor{string}{rgb}{.5,.15,.15}      % dunkelrot




%\usepackage{caption}
%\DeclareCaptionType{mycapequ}[][List of equations]
%\captionsetup[mycapequ]{labelformat=empty}

\begin{document}
\section{Funktionen}

\subsection{Bin ich drin?}
Pr�fe ob ein Wert im Intervall liegt. Ben�tigte Operatoren f�r Zahlen $<$, $>$ und $=$. Au�erdem and und or f�r die Verkn�pfung von boolschen bzw. Wahrheitswerten und if f�r die Auswertung.

\subsection{Berechne die Summe aller Zahlen von 1 bis n}
Die Umsetzung sollte mit einer Schleife passieren. Welche M�glichkeiten fallen dir noch ein?

\subsection{!murehtrhekreV: Gibt den Text r�ckw�rts aus!}
Als �bergabewert gibt es einen Text als string. Von diesem kann man mit Hilfe der funktion Length() die L�nge bestimmen. Au�erdem kann man mit dem Operator $[i]$ auf eine Buchstaben im Text an der Stelle i zugreifen. Beachte, dass strings bei Index 1 beginnen nicht bei 0 wie Arrays. Weiterhin wird der Operator $+$ ben�tigt um Texte zusammen zu setzen. 

\subsection{Zaehle alle Vokale!}
Pr�fe jeden Buchstaben im Text ob es ein Vokal ist und gib deren Anzahl zur�ck. Beachte Gro� und Kleinbuchstaben.

\subsection{Buchstabiere das Wort!}
Wandle das �bergebene Wort in eine durch Leerzeichen getrennte Aussprache seiner Buchstaben um. Aus dem Text "`abcdjky"' wird zum Beispiel "`a be ce jot ka ypsilon"'.

\subsection{Sage die Ziffern!}
Wandle die �bergebene Zahl in den entsprechenden Text bei der Aufz�hlung der Ziffern getrennt durch Leerzeichen um. Hilfreiche Operatoren sind div und mod sie geben jeweils den ganzzahligen Teil einer Division (div) und den Rest (mod) zur�ck. Aus 1234 wird z.B. "`eins zwei drei vier"'. Welches ist die beste Reihenfolge um die Zahl auseinander zu nehmen?

\subsection{Berechne die Summe!}
Eine Folge von Zahlen wird als Leerzeichen getrennter Text �bergeben. Trenne den Text und addiere die entstehenden Zahlen nach ihrer Umwandlung. �berlege dir zuerst ein m�glichst einfaches Vorgehen wie du den Text untersuchst. N�tzlich sind z.B. die Funktionen Pos und PosEx um einen bestimmten Text in einem Wert in einem Text zu finden.

\subsection{Welche Zahl in der Folge fehlt?}
Als �bergabewert gibt es eine Folge von Zahlen in Form eines Arrays. Die Zahlen sind sortiert und sollten sich immer um 1 unterscheiden. Gib die fehlende Zahl zur�ck falls in der Folge ein Sprung gr��er als 1 Auftritt. Gibt es Keinen, gib eine 0 zur�ck.

\subsection{Begr��e mich!}
Als �bergabewerte gibt es eine Tageszeit in Form eines Aufz�hlungstyps Tageszeit und einen Namen. Je nach Tageszeit soll eine andere Begr��ung als R�ckgabewert entstehen. 
\begin{tabular}{lcr}
  morgens & Guten Morgen \\
  mittags & Mahlzeit \\
	abends & Guten Abend \\
	nachts & Gute Nacht \\
\end{tabular}

Z.B. wird aus morgens und Bernd "`Guten Morgen, Bernd!"'. Benutze eine case Anweisung zur Unterscheidung der Tageszeit.

\subsection{Berechne die Hypotenuse!}
Als �bergabewerte gibt es die L�nge der beiden Seite am rechten Winkel eines Dreieck. Welche Funktionen ben�tigt man f�r Wurzeln und Quadrate von Zahlen?

\subsection{Sag's mir sp�ter.}
Gib den Text zur�ck der beim vorherigen Aufruf der Funktion �bergeben wurde. Gib beim ersten Aufruf einen leeren string zur�ck. Was ist die Schwierigkeit hieran?

\subsection{Sortiere die Liste}
Sortiere die Zahlen im �bergebenen Array von klein nach gro�. Ein einfacher Sortieralgorithmus ist Bubblesort informiere dich dazu und setze den Algorithmus um. Dieser Algorithmus muss mehrfach �ber die zu sortierenden Daten laufen. Was ist nach dem ersten Durchlauf passiert?

\end{document}